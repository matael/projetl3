% vim:ft=tex:
%
\documentclass[11pt]{article}
\usepackage[french]{babel}
\usepackage[utf8]{inputenc}
\usepackage{tikz}
\usepackage{amsmath}

\renewcommand{\phi}\varphi

\title{Modes dans un guide cyclindrique}
\author{C. Laurent \& M. Gaborit}
\date{Projet L3}

\begin{document}
\maketitle

On cherche à définir les modes dans un guide d'onde cylindrique.

\begin{figure}
\begin{tikzpicture}[>=latex,scale=0.6]
    % cercle
    \draw (0,0) circle (5);

    % axes
    \draw (-6,0) -- (6,0);
    \draw (0,-6) -- (0,6);
    \draw (0,0) circle (0.4);
    \draw[fill=black] (0,0) circle (0.1);


    % vecteur
    \draw[->] (0,0) -- (30:4) node[midway,left,above] {$r$};
    \draw (2,0) arc (0:30:2);
    \draw (15:2) node[right] {$\theta$};
    \draw[fill=black](30:4) circle (.1) node[right] {$M$};

    % Rayon
    \draw (0,0) -- (40:-6) node[midway,below,right] {$R$};
\end{tikzpicture}
\caption{\label{cyl1:fig} Coupe du guide}
\end{figure}

\subsection*{Position du problème}

On se place en coordonnées cylindriques (voir figure~\ref{cyl1:fig}, donc sur un domaine $D =
\{\forall(r,\theta)\in[0,1]\times[-\pi;\pi]\}$), en régime harmonique.

\paragraph{Equation de propagation} Ainsi, on cherchera à résoudre l'équation dite d'Helmholtz~\eqref{cyl1:helm} :

\begin{equation}
\Delta p  + k^2p = 0 \label{cyl1:helm}
\end{equation}

\paragraph{Laplacien} En coordonnées cyclindriques :

$$\Delta \bullet = \frac{1}{r}\frac{\partial}{\partial r}\left[r\frac{\partial\bullet}{\partial r}\right] +
\frac{1}{r^2}\frac{\partial^2\bullet}{\partial\theta^2}+\frac{\partial^2\bullet}{\partial z^2}$$

En coordonnées cylindriques, on a $p(t,r,\theta,z)$. On choisira de séparer les variables :

$$p(t,r,\theta,z) = \psi(r)\phi(\theta)e^{j(\omega t+k^{(z)}z)}$$

On réécrit donc~\eqref{cyl1:helm} :

\begin{equation*}
\begin{array}{c}
\frac{1}{r}\frac{\partial}{\partial r}\left[r\frac{\partial\psi}{\partial r}\right]\phi +
\frac{1}{r^2}\frac{\partial^2\phi}{\partial\theta^2}\psi+k^2\psi\phi = 0\\
%
\frac{1}{r\psi}\frac{\partial}{\partial r}\left[r\frac{\partial\psi}{\partial r}\right] +
\frac{1}{\phi r^2}\frac{\partial^2\phi}{\partial\theta^2}+k^2 = 0\\
%
\frac{1}{r\psi}\left[r\frac{\partial^2\psi}{\partial r^2}+\frac{\partial\psi}{\partial r}\right] +
\frac{1}{\phi r^2}\frac{\partial^2\phi}{\partial\theta^2}+k^2 = 0\\
%
\frac{r}{\psi}\left[r\frac{\partial^2\psi}{\partial r^2}+\frac{\partial\psi}{\partial r}\right] + k^2r^2 +
\frac{1}{\phi}\frac{\partial^2\phi}{\partial\theta^2} = 0\\
\end{array}
\end{equation*}

On a alors :
\begin{equation}
-\left\{\frac{r}{\psi}\left[r\frac{\partial^2\psi}{\partial r^2}+\frac{\partial\psi}{\partial r}\right] + k^2r^2\right\} =
\frac{1}{\phi}\frac{\partial^2\phi}{\partial\theta^2} = -\mu ~~;~~ \mu < 0\label{cyl1:sepvar}\\
\end{equation}

\paragraph{Condition de recollement} En $\theta=\pm\pi$ (aux limites du domaine d'étude) la fonction $\phi(\theta)$ doit
avoir une même valeur ainsi, nous avons la condition de recollement suivante :

\begin{equation}
\begin{cases}
    \phi(-\pi) = \phi(\pi)&\\
    \frac{\partial\phi(-\pi)}{\partial \theta} = \frac{\partial\phi(\pi)}{\partial \theta}&
\end{cases}\label{}
\end{equation}


\end{document}

